\documentclass{article} \usepackage[utf8]{inputenc}
\usepackage{xstring}
\usepackage{calc, xintexpr}
\usepackage{xfp}

% TeX docs at http://pgfplots.sourceforge.net/TeX-programming-notes.pdf

\title{Example of Variable Input}
\author{Cordelia David} \date{April 2019}
\input{input.tex}

% https://tex.stackexchange.com/a/7183/102826
% if is num, emphasize, else ignore.
\def\isnum#1{%
  \if!\ifnum9<1#1!\else_\fi
    \emph{#1}\else#1\fi}


\newcommand{\firstchar}[1]{%
    \StrLeft{#1}{1}[\firstletter]%
    \firstletter
}

% documentation: \getchar[hello]{5}   is 'h'
\def\getchar[#1]#2{%
\StrMid{#1}{#2}{\the\numexpr #2 + 0\relax}[\mychar]%
\mychar}


\begin{document}


\maketitle

\section{Introduction}

We had \Strawberries\ Strawberries for this year's harvest. Probably not enough.

% get first character of string hello
\getchar[hello]{1}
%\firstchar{\include{inputfile}}

% compute integer arithmetic
\inteval{3 - 2}

% compute integer arithmetic
% "Although \relax does nothing by itself, it is a safe command to stop expansion of another command"
\newcommand\three{3}
\the\numexpr \three - 2 + 5 \relax
% and to store it in a variable???? redirection and then reusing \three does not work. We need a new variable.
\def\tmp{\three}
\def\four{\the\numexpr \tmp - 2 + 5 }

% include the input file
\def\inputstring{\input{inputfile.tex}}
% and print it
\inputstring

% include the input file in a different way because the previous way fails when passed as argument to getchar
\newread\file
\openin\file=inputfile.tex
\read\file to \fileline
\closein\file

hello world 
% get second character of inputfile
\getchar[\three]{1}
\typeout{===>"three"|\three}

% Variables given by the task


% variables given by the task:
\newcount{\imgwidth}
\imgwidth=25
\newcount{\imgheigth}
\imgheigth=6

% compute size of a layer
\newcount{\layersize}
\layersize=\the\imgwidth
\multiply\layersize by \imgheigth \relax
\typeout{Size of a layer is \the\layersize}
I want to loop \the\layersize times for the first layer.

% store the index of the best layer. The best layer has the fewest digits. So initialize the counter to maximum possible + 1.
\newcount\bestlayer
\bestlayer=-1
\newcount\bestlayerzerocount
\bestlayerzerocount=\inteval{\layersize + 1}
\newcount\currentlayerzerocount
\newcount\currentlayer
\newcount\currentchar

% now for the first layer count the number of zeros
\currentlayer=0
\currentlayerzerocount=0
\currentchar=0
\newcount\digitctr
\digitctr=1 % count from 1, not from zero. So we count from 1 to \layersize
\loop
  \message{\the\digitctr}
  \advance \digitctr +1
  % assign current char
  %TODO: \currentchar=\getchar[\inputstring]{\the\digitctr}
  % check if zero
  \if 0\the\currentchar
    \advance \currentlayerzerocount 1
  \fi
\ifnum \digitctr<\inteval{\layersize + 1}
\repeat
% if there were very little zeros, we can update the best layer
\ifnum \currentlayerzerocount<\bestlayerzerocount
  \bestlayer=\currentlayer
  \bestlayerzerocount=\currentlayerzerocount
\fi
Best Layer Zero Count is \the\bestlayerzerocount

\getchar[\fileline]{2}


\end{document}