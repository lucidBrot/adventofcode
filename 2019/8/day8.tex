\documentclass{article} \usepackage[utf8]{inputenc}
\usepackage{xstring}
\usepackage{calc, xintexpr}
\usepackage{xfp}

\title{Example of Variable Input}
\author{Cordelia David} \date{April 2019}
\input{input.tex}

% https://tex.stackexchange.com/a/7183/102826
% if is num, emphasize, else ignore.
\def\isnum#1{%
  \if!\ifnum9<1#1!\else_\fi
    \emph{#1}\else#1\fi}


\newcommand{\firstchar}[1]{%
    \StrLeft{#1}{1}[\firstletter]%
    \firstletter
}

% documentation: \getchar[hello]{5}   is 'h'
\def\getchar[#1]#2{%
\StrMid{#1}{#2}{\the\numexpr #2 + 0\relax}[\mychar]%
\mychar}


\begin{document}


\maketitle

\section{Introduction}

We had \Strawberries\ Strawberries for this year's harvest. Probably not enough.

% get first character of string hello
\getchar[hello]{1}
%\firstchar{\include{inputfile}}

% compute integer arithmetic
\inteval{3 - 2}

% compute integer arithmetic
% "Although \relax does nothing by itself, it is a safe command to stop expansion of another command"
\newcommand\three{3}
\the\numexpr \three - 2 + 5 \relax
% and to store it in a variable???? redirection and then reusing \three does not work. We need a new variable.
\def\tmp{\three}
\def\four{\the\numexpr \tmp - 2 + 5 }

% include the input file
\newcommand\inputstring{\input{inputfile.tex}}
% and print it
\inputstring

hello world 
% get second character of inputfile
\getchar[\three]{2}
\typeout{===>"three"|\three}

% Variables given by the task


% Loop through first layer and count digits that are 0
% multiplication with a constant:
%\newcounter{imgwidth}{25}
%\newcounter{imgheigth}{6}
%\newcounter{digitctr}{
%2 * \value{imgheight}
%}

\newcount{\imgwidth}=25\relax
\newcount{\imgheigth}=6\relax
\newcount{\digitctr}
\digitctr=\imgwidth
\typeout{digitctr is \the\digitctr}
\multiply\digitctr by \imgheigth \relax
\typeout{digitctr is \the\digitctr}
I want to loop \digitctr times for the first layer.

\end{document}